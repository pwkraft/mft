\documentclass[12pt]{article}
\usepackage[margin = 1in]{geometry}
\usepackage[USenglish]{babel}
\usepackage{natbib}
\usepackage{multirow}
\usepackage{graphicx}
\usepackage{fancyhdr}
\usepackage{setspace}
\usepackage{verbatim}
\usepackage{booktabs}
\usepackage{amsmath}
\usepackage{lscape}
\usepackage{dcolumn}
\usepackage{subcaption}
\usepackage[title]{appendix}
\usepackage{xcolor}
\usepackage{todonotes}
\usepackage{titletoc}
\usepackage{longtable}
\usepackage[colorlinks=true,citecolor=red!50!black,urlcolor=blue!50!black,linkcolor=red!50!black]{hyperref}

\author{Patrick W. Kraft\footnote{Ph.D. Candidate, Stony Brook University, \href{mailto:patrick.kraft@stonybrook.edu}{patrick.kraft@stonybrook.edu}.
%I am indebted to Jennifer Jerit, Yanna Krupnikov, Jason Barabas, Stanley Feldman, Fridolin Linder, Hannah Nam, Scott Clifford, Peter DeScioli, Scott Bokemper, and participants of the Political Science Graduate Student Colloquium at Stony Brook University as well as participants at the panel of the 2015 Annual Meeting of the Midwest Political Science Association for helpful comments on earlier drafts of this paper. Furthermore, I thank Leonie Huddie for sharing the data for the telephone survey replication.
}}
%\date{}


\title{Measuring Morality in Political Attitude Expression\footnote{An earlier version of this paper was presented at the 73rd Annual Conference of the Midwest Political Science Association, April 16-19, 2015. The manuscript and code are available on GitHub: \url{https://github.com/pwkraft/mft}.}}



\begin{document}
\maketitle
\doublespacing
\thispagestyle{empty}

\begin{abstract}
This study explores whether and how individuals evoke moral considerations when discussing their political beliefs. Analyzing open-ended responses in the 2012 American National Election Study using a previously validated dictionary, I find systematic ideological differences in moral reasoning---even when respondents are not explicitly asked about morality. The study proceeds to show that the reliance on moral considerations in attitude expression is conditional on the moral content of individual media environments.

%\vspace{\baselineskip}
%\noindent \textbf{Keywords:} Moral Foundations Theory, Open-ended Survey Responses, Individual Media Environments
\end{abstract}
\newpage
\setcounter{page}{1}

Increasing levels of polarization has renewed scholarly interest in the psychological and attitudinal differences between liberals and conservatives \citep{jost2006end}. One such area of research focuses on the moral underpinnings of ideology. According to \textit{Moral Foundations Theory} (MFT), moral thinking is organized by five innate intuitions: harm/care, fairness/reciprocity, ingroup/loyalty, authority/respect, and purity/sanctity \citep{haidt2012righteous,graham2013moral}. Liberals and conservatives differ in their relative emphasis on these foundations, with liberals prioritizing the foundations of harm/care and fairness/reciprocity, and conservatives endorsing all five dimensions equally \citep{graham2009liberals}.

% DIRECTLY get into media environment and framing of moral arguments here...
% CHECK Feinberg citations

A series of recent studies shows that the moral foundations influence issue preferences \citep{koleva2012tracing, kertzer2014moral}, candidate trait evaluations \citep{clifford2014linking}, and vote choice \citep{iyer2010beyond, franks2015using}. Research further suggests that elite communications can affect the relevance of individual moral foundations for attitude formation and attitude change \citep[e.g.][]{clifford2013words,clifford2015concerns,day2014shifting,feinberg2013moral}. For the most part these studies measured moral reasoning with the Moral Foundations Questionnaire (MFQ) to measure moral reasoning, which explicitly asks respondents to judge the importance of considerations related to the five foundations \citep[e.g.][]{graham2011mapping}. However, by explicitly raising moral arguments surveys, we presuppose that the connection between moral values and politics indeed manifests itself in individual reasoning. Some scholars have criticized the MFQ because it does not ask people to make moral judgments per se \citep[e.g.][]{clifford2015moral}. Indeed, \citet[1031]{graham2009liberals} describe the reports on moral relevance as ``self-theories of moral judgment,'' rather than direct measures of judgment itself.

This study addresses this gap by examining whether people utilize the foundations in a more unobtrusive context (i.e., without being prompted by the language of a questionnaire). Using a moral dictionary validated in previous studies \citep[e.g.,][]{graham2009liberals}, I analyze individual verbatim responses to open-ended questions about political attitudes and preferences. Using open-ended items this way allows for a direct investigation of moral reasoning where the potential connection between morality and politics is not induced or facilitated by design. Insofar as moral intuitions play a role in political attitude expression, citizens should rely on the moral foundations when discussing their opinion about political actors, even if not explicitly asked to do so. 

The first step of the analyses focuses on the replication of previous findings connected to MFT and ideology using open-ended survey responses in the 2012 ANES. Consistent with MFT, the results reveal systematic differences between liberals and conservatives in the reliance on specific moral considerations even without being cued to think about morality. Furthermore, these differences in moral reasoning influence candidate preferences and vote choice---even after controlling for a person's party identification. Integrating a large-scale content analysis of individual media environments, the analyses proceed to show that individuals who are exposed to moral rhetoric in political news are more likely to rely on moral considerations when discussing their political beliefs. This study contributes to the literature by demonstrating how open-ended survey responses can be utilized to investigate the underpinnings of political reasoning.


\section*{Measuring Moral Reasoning in Open-Ended Responses}

Using the moral foundations dictionary created by \citet{graham2009liberals}, I identify references to specific moral considerations when respondents discuss what they like and dislike about political parties and candidates.\footnote{See Appendix~\ref{app:dict} for the full content of the dictionary.} Other studies have relied on (variations of) this dictionary to investigate moral considerations in elite communication \citep[e.g. in news media coverage about stem cell research,][]{clifford2013words}, but to date no research has examined verbatim attitude expressions in surveys.

Based on the terms signaling each foundation in the dictionary, any document can be scored according to its emphasis on the respective moral dimension. Conventional dictionary-based methods usually consist of the proportion of signal word occurrences in each document \citep[e.g.][]{graham2009liberals}. However, some terms in the dictionary might be problematic when applied to verbatim survey responses. In particular, when respondents describe their attitudes towards political actors, certain words might be too ubiquitous to be regarded as an unambiguous indicator for specific moral considerations. For example, the moral foundations dictionary includes ``leader'' as a signal word for the authority/respect dimension. However, many respondents may be inclined to describe the qualities of presidential candidates as \textit{leaders}, irrespective of moral considerations related to authority?

One way to address this problem using conventional methods would be to revise the dictionary and eliminate words that are deemed problematic. Yet such revisions could be arbitrary and leave a lot of discretion to the researcher. Drawing on techniques developed in the field of information retrieval, I proposes an alternative approach. If it is the case that ``leader'' represents a term that is commonly used to describe presidential candidates (irrespective of moral considerations), it should appear more frequently in open-ended responses across individuals. Terms that are used by almost all respondents therefore provide less information about differences in their (moral) reasoning than terms that only occur in few responses.  Stated differently, if a specific moral word is mentioned by a large majority of respondents, it is more likely that the term can be used in multiple contexts and is not necessarily unique to the moral domain. In this study, MFT scores are computed for a foundation by weighting each term in the dictionary according to its ubiquity across documents, which serves as a proxy for the term's discriminative information:
\begin{equation}\label{eq:tfidf}
\text{MFT}_{if} = \dfrac{1}{W_i} \sum_{t \in \mathcal{D}_f} \left[ w_{it} * \log_{10}\left( \dfrac{N}{n_t+1}\right) \right],
\end{equation}
where $\text{MFT}_{if}$ denotes the score of document $i$ for foundation $f$, $W_i$ is the total number of words in document $i$, $t$ indicates a term in the set of signal terms in foundation dictionary $\mathcal{D}_f$, and $w_{it}$ denotes the number of occurrences of term $t$ in document $i$. Furthermore, $N$ denotes the total number of documents, and $n_t$ is the number of documents in which the term $t$ appears. The weight represents the inverse of the proportion of documents in which the target term appears.\footnote{This specification is usually referred to as tf-idf weighting and is commonly used in quantitative text analyses. The acronym tf-idf stands for ``term frequency - inverse document frequency,'' which refers to the rationale that the frequency of specific terms are weighted by the inverse of the frequency of occurrence across documents. See \citet[ch. 6]{manning2008introduction} for an introduction.} As such, terms that are ubiquitous across the entire corpus receive a lower weight, and terms that appear in only few documents receive a higher weight. The denominator in equation \eqref{eq:tfidf} includes $+1$ to ensure that it does not equal zero if a dictionary term does not appear in any of the documents.

In the analyses presented here, each document is an individual's verbatim response to a set of open-ended questions. As such, a respondent's MFT score for foundation $f$ is the weighted proportion of words in the response that signal the respective foundation. The score has a lower bound of 0 (document does not contain any dictionary terms) and is independent of document length (since it is based on relative occurrences). Higher scores imply larger proportions of dictionary terms in a document. Most importantly, however, words that appear in nearly all open-ended remarks affect MFT scores less than the words mentioned only by few respondents because ubiquitous words convey less information about differences across individuals. Overall, the MFT score provides a correction for potential distortions due to suboptimal terms in the dictionary while leaving its exact content outside the researcher's discretion. Since nominal values of the MFT score above zero do not have a clear substantive interpretation, they are rescaled to unit variance.


\section*{Results}

The analyses are based on the 2012 ANES. The primary dependent variables (i.e., the MFT scores described above) are based on open-ended questions in which respondents were asked to report what they \textit{liked} and \textit{disliked} about either presidential candidate as well as the Republican and Democratic parties. The responses were aggregated for each individual and pre-processed by correcting spelling errors using an implementation of the Aspell spell checking algorithm in \texttt{R} (\url{www.aspell.net}).\footnote{Please refer to the appendix for further information on the data and recoding. The dimension of purity/sanctity was omitted from the analyses of open-ended responses due to its low general prevalence in individual attitude expressions.} Due to the fact that the MFT scores are bound at zero (i.e., none of the words in the dictionary appear in the response), individual response patterns are modeled via Tobit regressions for each of the moral foundations under consideration. I decompose the estimates into the effect on the probability of mentioning a specific foundation \textit{at all} as well as the degree of emphasis on the foundation given that it was mentioned by a respondent \citep[c.f.][]{mcdonald1980uses}. All models control for education, logged overall response length, and the Wordsum vocabulary score, which should account for potential confounding factors related to the respondents' eloquence when discussing their political attitudes.

% MENTION: why purity was omitted


\subsection*{Ideological Differences in Moral Reasoning}

I begin by estimating a set of Tobit regressions using ideology (and the control variables discussed above) to predict the individual MFT score for each of the moral foundations.\footnote{The full estimates for this and all subsequent models are presented in Appendix~\ref{app:tables}.} To reiterate, the MFT score measures the weighted proportion of moral foundation terms in an open-ended response. Figure~\ref{fig:tobit_ideol} compares liberals and conservatives while holding all other variables constant at their respective means. The effects of ideology (liberal - conservative) are decomposed into two parts: the left panel displays the change in probability of mentioning a specific foundation at all (i.e., probability of the MFT score to be larger than zero), whereas the right panel displays the expected change in the degree of emphasis on a foundation given that it was mentioned  (i.e., the change in the MFT score given that it is larger than zero, measured in standard deviations).

\begin{figure}[ht]\centering
\includegraphics{../calc/fig/tobit_ideol.pdf}
\caption{Difference between liberals and conservatives in the probability of mentioning each moral foundation (left panel), and in the MFT score given that the foundation was mentioned (right panel), holding all other control variables at their respective means (along with 95\% confidence intervals). Control variables include church attendance, education, age, sex, race, survey mode, response length, and the Wordsum vocabulary score. Full model results are displayed in the appendix, Table~\ref{tab:tobit_ideol}.
}\label{fig:tobit_ideol}
\end{figure}

Positive values denote a higher probability of mentioning the respective moral foundation (left panel) or a higher MFT score (right panel) of a response among individuals who identified as liberals, while negative values indicate a higher probability/higher score among conservatives. The effects are consistent with the the expectations of MFT for three out of four moral foundations. Liberals are significantly more likely to mention the foundations of harm/care and fairness/reciprocity. More specifically, liberals were approximately 6 percentage points more likely than conservatives to reference these two foundations. Furthermore, given that respondents mention these two foundations at all, liberals emphasize it more than conservatives when evaluating political parties and candidates. The MFT score for the harm/care foundation is about 0.07 standard deviations higher among liberals than conservatives. The effect is slightly larger for the fairness/reciprocity dimension. Conversely, being conservative increases the MFT score for the foundation of ingroup/loyalty by about 0.09 standard deviations. There are no significant differences between liberals and conservative on the authority/respect dimension. The dimension of purity/sanctity was omitted due to its low general prevalence in individual attitude expressions.

% MAYBE add this bit to the conclusion
%Taken together, the results are largely consistent with previous findings in the literature on MFT. Individuals evoke moral considerations when evaluating political parties and candidates without being explicitly asked about morality. Furthermore, there are systematic differences between liberals and conservatives in their reliance on binding and individualizing foundations \citep[see][for similar ideological differences when analyzing the content of life-narrative interviews]{mcadams2008family}. However, the fact that the authority/respect foundation showed insignificant patterns could suggest that the public's reliance on moral foundations may be more context-specific than previously thought.


\subsection*{The Political Relevance of Moral Reasoning}

A skeptical reader may argue that even if the ideological patterns are consistent with MFT, the expression of moral considerations might not be as strongly related to other forms of political behavior (e.g. vote choice) as latent moral foundations measured by the MFQ. To address this concern, Figure~\ref{fig:logit_vote} presents the changes in expected probabilities of voting for the Democratic (vs. the Republican) presidential candidate in the 2012 election for individuals emphasizing the moral foundations in their open-ended responses. The estimated probabilities are based on logit models including MFT scores for each moral foundation as independent variables (as well as  the remaining controls), which were held constant at their mean values when computing expected values. Individuals who emphasized moral considerations related to the harm/care and fairness/reciprocity foundations are more likely to vote for Barack Obama than for Mitt Romney. Respondents who emphasized the ingroup/loyalty foundation, on the other hand, were less likely to vote for Obama.\footnote{A similar pattern can be observed in an analysis of feeling thermometers towards both parties and candidates. Please refer to the appendix for details.}

\begin{figure}[ht]\centering
\includegraphics[scale=.9]{../calc/fig/logit_vote.pdf}
\caption{Change in predicted probabilities to vote for the Democratic rather than Republican candidate when MFT score is increased from its minimum (no overlap between dictionary and response) by one standard deviation, holding all other control variables constant at their respective means (along with 95\% confidence intervals). Control variables include party identification, church attendance, education, age, sex, race, survey mode, response length, and the Wordsum vocabulary score. Full model results are displayed in the appendix, Table~\ref{tab:logit_vote}.
}\label{fig:logit_vote}
\end{figure}

The effects on vote choice might not seem large, but bear in mind that the measure of moral reasoning is based solely on the content of open-ended responses in which respondents were \textit{not} explicitly asked about morality. Yet, the moral considerations evoked by respondents are powerfully related to party and candidate evaluations as well as vote choice. Overall, the analyses show that people's open-ended comments about both candidates and both parties are imbued with moral content and that these comments relate to political judgments in the manner predicted by MFT.


\subsection*{Media Content and Exposure to Moral Rhetoric}

Next, I investigate whether the general reliance on moral considerations is a product of exposure to moralized political discourse. For each individual, I compute the sum of MFT scores to measure the general tendency to emphasize \textit{any} moral foundation. The main independent variable captures moralization of media environments based on a content analyses of media sources used by each individual. The 2012 ANES includes a large array of items indicating whether respondents regularly consumed various news outlets. For all media sources available, I downloaded the content of the coverage on either presidential candidates during the survey field period in the last month of the campaign (October 2012) from Lexis-Nexis and coded their emphasis on moral foundations using the same approach as for open-ended survey responses. Similar to the respondent MFT scores, the media scores were summed over all foundations to capture the general degree of moralization.\footnote{In total, I retrieved the content of 28 media sources, such as the New York Times (print and online), CNN.com, or various Fox News Programs. See Figure~\ref{fig:media_desc} in the Appendix for a more detailed overview of the media outlets and their respective MFT scores.} 

Based on the coded content for each media source, I create a measure that represents the extent to which each individual's media environment emphasized moral considerations. For each respondent in the ANES, I select the media sources he or she reported to watch/read regularly and computed the sum of the sources' MFT scores. Using this approach, I can analyze whether individuals who rely on media sources that use more moralized reporting were also more likely to emphasize moral arguments in their open-ended responses.

\begin{figure}[h]\centering
\includegraphics{../calc/fig/tobit_media.pdf}
\caption{Effect of MFT content in individual media environments on the probability of mentioning any moral foundation (left panel), and on the summed MFT score given that any foundation was mentioned (right panel), holding all other control variables at their respective means (along with 95\% confidence intervals). Control variables include general media exposure, political knowledge, political discussion frequency, church attendance, education, age, sex, race, survey mode, response length, and the Wordsum vocabulary score. Full model results are displayed in the appendix, Table~\ref{tab:tobit_cont}.
}\label{fig:tobit_media}
\end{figure}

The results are presented in Figure~\ref{fig:tobit_media}. Estimates are based on Tobit models that take into account the censoring of the moral reasoning measure and effects are decomposed into the probability of mentioning any moral foundation (left panel) as well as the emphasis on morality, given that any foundation was mentioned (right panel). Individuals who are exposed to media sources that report on the campaign in a more moralized manner put a stronger emphasis on moral arguments in their open-ended responses describing their political attitudes about the parties and candidates. Thus, citizens learn to embed moral reasoning in their political evaluations by adopting moral arguments from their media environment.

% COMMENT: mention that this result is principally consistent with selective exposure


\subsection*{Robustness Checks}

To this point, the analyses assume that the dictionary-based approach for open-ended responses captures the theoretical concept of interest---\textit{moral} reasoning. Yet, the terms in the dictionary may be recovering other (i.e., non-moral) differences in word choice between liberals and conservatives when discussing their attitudes towards parties and candidates in the 2012 U.S. Presidential election. For example, one prominent issue in the election was the Affordable Care Act, which might increase the likelihood of Democrats mentioning the term ``care'' and thereby increasing the emphasis on the harm/care foundation irrespective of underlying moral considerations. As such, the observed differences might be an artifact due to the nature of the questions under considerations as well as the specific political context of the presidential campaign.

To address this concern, I replicated the main model focusing on ideological differences in moral foundations using open-ended responses from a survey administered in a different political context. The survey was conducted via telephone with 594 adults aged 18 or older between early January, 2001 and July, 2003. The telephone numbers were a random-digit-dial (RDD) sample drawn from residents within a 25 mile radius of a large northeastern state university. As such, the survey was not conducted during a  major presidential election campaign. Furthermore, the survey uses a different set of open-ended items. Rather than asking about attitudes towards presidential candidates and both major parties, respondents were asked to describe liberals and conservatives as well as their respective beliefs in general. The coding and analyses are equivalent to those for Figure~\ref{fig:tobit_ideol}, although the survey did not contain the Wordsum scores included in the main analyses. The results are displayed in Figure~\ref{fig:tobit_ideol_lisurvey}.

\begin{figure}[ht]\centering
\includegraphics{../calc/fig/tobit_ideol_lisurvey.pdf}
\caption{Replication of main model (c.f., Figure~\ref{fig:tobit_ideol}) using RDD sample from residents within a 25 mile radius of a large northeastern  state university. Figure displays difference between liberals and conservatives in the probability of mentioning each moral foundation (left panel), and in the MFT score given that the foundation was mentioned (right panel), holding all other control variables at their respective means (along with 95\% confidence intervals). Control variables include church attendance, education, age, sex, race, and response length. Full model results are displayed in the appendix, Table~\ref{tab:tobit_ideol_lisurvey}
}\label{fig:tobit_ideol_lisurvey}
\end{figure}

The patterns are consistent with previous results. Liberals are more likely to emphasize the foundations of harm/care and fairness/reciprocity. The result for the ingroup/loyalty dimension, however, do not reach conventional levels of statistical significance. Additional analyses reveal that the ideological differences in moral reasoning are mostly due to the fact that individuals who identify as liberals emphasize the foundations of harm/care and fairness/reciprocity more strongly than conservatives when describing their ingroup (i.e., other liberals and their beliefs), while conservatives emphasize the foundation of ingroup/loyalty more strongly than liberals when describing their ingroup (results available upon request). The fact that the same basic ideological pattern can be recovered in a survey that was conducted in a different political context (non-election period, Republican administration), employed a different survey mode (phone interview), and relied on a different set of open-ended survey questions (asking about liberals and conservatives and their respective beliefs), provides additional evidence that the MFT dictionary recovers basic moral considerations in political reasoning.


\section*{Discussion}

% ADD:
% The present study is the first to investigate moral reasoning by examining individual verbatim expressions of political attitudes and preferences.
% Specifically, I introduce methods to improve conventional dictionary-based approaches for the analysis of open-ended responses and showcase the integration of media content analyses to trace the influence of exposure to political discourse on individual response behavior.

This study utilized open-ended survey responses to investigate morality in individual attitude expression. The analyses of open-ended survey responses is valuable in this context because it allows researchers to evaluate whether citizens make references to moral considerations in a political context that does not induce an explicit connection to morality. As such, it can be directly investigated when and how ideological differences in the emphasis of moral foundations manifest themselves in individual reasoning about political actors. More generally, open-ended survey responses provide a promising and still largely neglected data source to investigate the determinants and structure of ideology and political reasoning. In particular, scholars can directly assess moral reasoning in surveys that do not contain the MFQ or related measures, simply by relying on open-ended survey responses. More broadly, focusing on open-ended measures provides new opportunities to study the role of morality in day-to-day political reasoning.

The empirical analyses presented here extend and qualify previous research on moral foundations and ideology. The results showed systematic patterns in the emphasis on moral considerations among liberals and conservatives consistent with MFT for three out of four foundations. Liberals are more likely to mention considerations related to harm/care and fairness/reciprocity when discussing their political preferences, whereas conservatives are more likely to emphasize the moral foundation of ingroup/loyalty. The second part of the analyses focused on the political relevance of moral reasoning as conceptualized by open-ended survey responses. Here, the results revealed consistent relationships between individual moral foundations and voting behavior, which showed that moral reasoning (measured via open-ended survey responses) is a politically meaningful and influential concept. Lastly, exposure to moralized political discourse increase the reliance on moral considerations. 

The present study therefore reaffirms the importance of moral reasoning in politics but also reveals its potential conditionality. Ultimately, a deeper understanding of the role of morality in politics necessitates further analyses of the broader political context that shapes individual information environments (e.g., how the endorsement of moral foundations varies over time and across campaigns). Such an investigation would further illuminate how exposure to political discourse fosters ideological differences in moral reasoning. In times of growing partisan polarization, a better understanding of the antecedents of this ideological divide is essential.

%\clearpage
\bibliographystyle{/data/Dropbox/Uni/Lit/apsr2006}
\bibliography{/data/Dropbox/Uni/Lit/Literature}

%\clearpage\footnotesize\singlespacing
%\setcounter{page}{1}
%\appendices
%\appendixpage
%\renewcommand\thesubsection{\Roman{subsection}}
%\begin{flushleft}
%Online appendices for manuscript: \\
%``Morals Matter, But Not For Everyone: The Conditionality of Moral Foundations in Political Reasoning''
%
%\startcontents[sections]
%\printcontents[sections]{l}{1}{\setcounter{tocdepth}{2}}
%\clearpage
%
%\section{Moral Foundations Dictionary}\label{app:dict}
%\textit{Sources:}\\
%\citet{graham2009liberals}, as well as \url{http://www.moralfoundations.org/}
%\vspace{.5cm}
%
%\textit{Note:}\\
%Words with (*) indicate that the word stem rather than the exact word was matched in the open-ended survey responses.
%\vspace{.5cm}
%
%\textbf{Harm:}\\
%safe*, peace*, compassion*, empath*, sympath*, care, caring, protect*, shield, shelter, amity, secur*, benefit*, defen*, guard*, preserve, harm*, suffer*, war, wars, warl*, warring, fight*, violen*, hurt*, kill, kills, killer*, killed, killing, endanger*, cruel*, brutal*, abuse*, damag*, ruin*, ravage, detriment*, crush*, attack*, annihilate*, destroy, stomp, abandon*, spurn, impair, exploit, exploits, exploited, exploiting, wound*
%\vspace{.5cm}
%
%\textbf{Fairness:}\\
%fair, fairly, fairness, fair*, fairmind*, fairplay, equal*, justice, justness, justifi*, reciproc*, impartial*, egalitar*, rights, equity, evenness, equivalent, unbias*, tolerant, equable, balance*, homologous, unprejudice*, reasonable, constant, honest*, unfair*, unequal*, bias*, unjust*, injust*, bigot*, discriminat*, disproportion*, inequitable, prejud*, dishonest, unscrupulous, dissociate, preference, favoritism, segregat*, exclusion, exclud*
%\vspace{.5cm}
%
%\textbf{Ingroup:}\\
%together, nation*, homeland*, family, families, familial, group, loyal*, patriot*, communal, commune*, communit*, communis*, comrad*, cadre, collectiv*, joint, unison, unite*, fellow*, guild, solidarity, devot*, member, cliqu*, cohort, ally, insider, foreign*, enem*, betray*, treason*, traitor*, treacher*, disloyal*, individual*, apostasy, apostate, deserted, deserter*, deserting, deceiv*, jilt*, imposter, miscreant, spy, sequester, renegade, terroris*, immigra*
%\vspace{.5cm}
%
%\textbf{Authority:}\\
%obey*, obedien*, duty, law, lawful*, legal*, duti*, honor*, respect, respectful*, respected, respects, order*, father*, mother, motherl*, mothering, mothers, tradition*, hierarch*, authorit*, permit, permission, status*, rank*, leader*, class, bourgeoisie, caste*, position, complian*, command, supremacy, control, submi*, allegian*, serve, abide, defere*, defer, revere*, venerat*, comply, defian*, rebel*, dissent*, subver*, disrespect*, disobe*, sediti*, agitat*, insubordinat*, illegal*, lawless*, insurgent, mutinous, defy*, dissident, unfaithful, alienate, defector, heretic*, nonconformist, oppose, protest, refuse, denounce, remonstrate, riot*, obstruct
%\vspace{.5cm}
%
%\textbf{Purity:}\\
%piety, pious, purity, pure*, clean*, steril*, sacred*, chast*, holy, holiness, saint*, wholesome*, celiba*, abstention, virgin, virgins, virginity, virginal, austerity, integrity, modesty, abstinen*, abstemiousness, upright, limpid, unadulterated, maiden, virtuous, refined, intemperate, decen*, immaculate, innocent, pristine, humble, disgust*, deprav*, disease*, unclean*, contagio*, indecen*, sin, sinful*, sinner*, sins, sinned, sinning, slut*, whore, dirt*, impiety, impious, profan*, gross, repuls*, sick*, promiscu*, lewd*, adulter*, debauche*, defile*, tramp, prostitut*, unchaste, wanton, profligate, filth*, trashy, obscen*, lax, taint*, stain*, tarnish*, debase*, desecrat*, wicked*, blemish, exploitat*, pervert, wretched*
%\vspace{.5cm}
%
%%\textbf{General:}\\
%%righteous*, moral*, ethic*, value*, upstanding, good, goodness, principle*, blameless, exemplary, lesson, canon, doctrine, noble, worth*, ideal*, praiseworthy, commendable, character, proper, laudable, correct, wrong*, evil, immoral*, bad, offend*, offensive*, transgress*, honest*, lawful*, legal*, piety, pious, wholesome*, integrity, upright, decen*, indecen*, wicked*, wretched*
%
%\end{flushleft}
%
%\renewcommand\thefigure{\thesection.\arabic{figure}}
%\renewcommand\thetable{\thesection.\arabic{table}}
%\setcounter{figure}{0}
%\setcounter{table}{0}
%
%\begin{center}
%\begin{longtable}{lp{1.5cm}p{5.5cm}p{5.5cm}}
%\caption[Open-Ended Responses]{Sample of open-ended responses in the 2012 American National Election Study. Responses were selected if their length was within 10 words of average responses ($\sim75$ words) and if they scored high on one of the moral foundations (see first column). The second and third column display the item category and the raw response. The last column displays the processed response highlighting all signal words for the respective foundation.}\label{tab:sample} \\
%
%\hline
%	\textbf{Foundation} & \textbf{Variable} & Raw Response & Processed Response \\ \hline \endfirsthead
%	
%	\multicolumn{4}{c}{{\tablename\ \thetable{} -- continued from previous page}} \\
%	\hline Foundation & Variable & Raw & Processed \\ \hline \endhead
%	
%	\hline \multicolumn{4}{r}{{Continued on next page}} \\	\endfoot
%	
%	\hline	\endlastfoot
%	
%	Harm & Obama (like) & supports ending war, supports affordable health care for all, supports the preservation of medicare and social security, looking into energy conservation to preserve our planet for future generations, initiatives to promote education and job growth and much more. & \multirow{8}{5.5cm}{supports ending \textit{war} supports affordable health \textit{care} for all supports the preservation of medicare and social \textit{secur} looking into energy conservation to \textit{preserve} our planet for future generations initiatives to promote education and job growth and much more imposing a fine if someone does not get a health \textit{care} plan he supports a strong military anti same sex marriage anti women s choice for abortion not supportive of health \textit{care} reform act not sensitive to the needs of the very poor and immigra} \\
%		 & Obama (dislike) & imposing a fine if someone does not get a health care plan \\
%		 & Romney (like) & he supports a strong military \\
%		 & Romney (dislike) & anti same sex marriage, anti women's choice for abortion, not supportive of health care reform act, not sensitive to the needs of the very poor and immigrants \\
%		 & Dems (like) & -1 Inapplicable \\
%		 & Dems (dislike) & -1 Inapplicable \\
%		 & Reps (like) & -1 Inapplicable \\
%		 & Reps (dislike) & -1 Inapplicable \\ \hline
%	
%	Fairness & Obama (like) & people rights, economy, taxes for working people, understanding of international problems & \multirow{8}{5.5cm}{people \textit{rights} economy taxes for working people understanding of international problems abortion \textit{rights} women \textit{rights} tax breaks for the rich military hawk rude and condescending to president obama economy women s \textit{rights} gay \textit{rights} health care tax plan for working class international strategy sometimes they do not fight hard enough against the republicans racist elitist trying to enrich the rich even more by hurt working people international relations health care women \textit{rights} gay \textit{rights}} \\
%	 & Obama (dislike) & -1 Inapplicable \\
%	 & Romney (like) & -1 Inapplicable \\
%	 & Romney (dislike) & abortion rights, women rights, tax breaks for the rich, military hawk, rude and condescending to President Obama \\
%	 & Dems (like) & economy, women's rights, gay rights, health care, tax plan for working class, international strategy. \\
%	 & Dems (dislike) & sometimes they do not fight hard enough against the republicans. \\
%	 & Reps (like) & -1 Inapplicable \\
%	 & Reps (dislike) & racist, elitist, trying to enrich the rich even more by hurting working people, international relations, health care, women rights, gay rights \\ \hline	
%	
%	Ingroup & Obama (like) &  & \multirow{8}{5.5cm}{he dozen t do enough about keeping us safe from our \textit{foreign} \textit{enem} he s too iffy about isle too favorable about homosexuality abortion like his close \textit{family} ties good ideas about keeping us safe from \textit{foreign} countries strong on israel they support same sex marriage they won t bend i like the people that are their leader i vie nevier been disappointed in their positions on most things} \\
%	 & Obama (dislike) & He doesn't do enough about keeping us safe from our foreign enemies; he's too iffy about Isael; too favorable about homosexuallity, abortion. \\
%	 & Romney (like) & Like his close family ties; good ideas about keeping us safe from foreign countries; strong on Israel; \\
%	 & Romney (dislike) &  \\
%	 & Dems (like) &  \\
%	 & Dems (dislike) & They support same sex marriage; they won't bend \\
%	 & Reps (like) & I like the people that are their leaders; I've never been disappointed in their positions on most things. \\
%	 & Reps (dislike) &  \\ \hline
%	 
%	 Authority & Obama (like) & competent, intelligent, but not strong in protecting US border, seriously dealing with illegal immigrant not rewarding for breaking the law. also, health care bill have me a little concern & \multirow{8}{5.5cm}{competent intelligent but not strong in protect us border seriously dealing with \textit{illegal} immigra not rewarding for breaking the \textit{law} also health care bill have me a little concern lack of strong laws dealing with \textit{illegal} aliens and us borders strong and firm dealing with foreign countries ie middle east china mexico the health care not too comfortable with what i am hearing about it} \\
%	 	 & Obama (dislike) & lack of strong laws dealing with illegal aliens and US borders, strong and firm dealing with foreign countries ie middle east, china, mexico; the health care--not too comfortable with what I am hearing about it. \\
%	 	 & Romney (like) & -1 Inapplicable \\
%	 	 & Romney (dislike) & -1 Inapplicable \\
%	 	 & Dems (like) & -1 Inapplicable \\
%	 	 & Dems (dislike) & -1 Inapplicable \\
%	 	 & Reps (like) & -1 Inapplicable \\
%	 	 & Reps (dislike) & -1 Inapplicable \\
%\end{longtable}
%\end{center}
%
%
%\section{Data, Variables, and Model Specification}
%
%
%The 2012 ANES which contains two representative cross-sectional samples. One sample was conducted by computer assisted face-to-face interviews while the other sample is based on an internet panel group. Both samples are pooled in the analyses. While each consisted of a pre-election and a post-election wave, most items described below are drawn from the pre-election wave.\footnote{The open-ended items were included only in the pre-election wave. Accordingly, wherever possible, the set of explanatory variables was limited to the pre-election wave.} 
%
%Respondents were not included in the analysis if they failed to provide an answer to all open-ended items, or if the interview language was Spanish. Table~\ref{tab:app_mis} in the appendix provides an overview of the number of omitted cases. About 4\% of the interviews were held in Spanish and about 7\% of the respondents did not provide any open-ended response. Furthermore, Figure~\ref{fig:appB2num} in the appendix displays histograms of the length of the respondents' answers to all open-ended items. On average, the collection of all open-ended responses consists of about 75 words for each individual. Table~\ref{tab:sample} in the appendix additionally provides a sample of average-length responses that scored high on each of the moral foundations to illustrate how responses were processed.
%
%
%The key independent variable used to predict the emphasis on each of the moral foundations, is \textit{political ideology}. Respondents were asked to place themselves on a seven-point scale ranging from extremely liberal to extremely conservative, which was transformed into dichotomous indicators for respondents who identified as liberals, conservatives, or moderates. Additional control variables included in the analyses are \textit{church attendance}, \textit{education} (college degree), \textit{age}, \textit{sex}, \textit{race} (African American), survey mode (online vs. offline), as well as the overall length of the individual responses in the open-ended questions (\textit{measured as logged number of words}). Furthermore, the 2012 ANES included the \textit{Wordsum} vocabulary test as a measure of literacy and verbal skill. It consists of a series of items asking respondents to choose a term that is closest to a target word. The Wordsum score consists of an additive index of correct responses in ten individual trials. The inclusion of the length of individual responses and the Wordsum score as control variables should account for potential confounding factors such as general effects of increased political literacy on the complexity of open-ended responses.
%
%In order to examine the relevance and consequences of moral reasoning measured through open-ended responses, the MFT scores for each of the moral foundations are used as independent variables to predict political outcomes. The dependent variables considered here are \textit{candidate} and \textit{party evaluations} (measured as the respective feeling thermometer differentials), as well as \textit{voting behavior} (measured as a dichotomous indicator of vote choice for the Democratic rather than the Republican Presidential candidate reported in the post-election wave). In addition to the controls discussed previously, these analyses include measures of \textit{party identification}, which were recoded similarly to ideology.
%
%The last set of analyses investigates whether the expression of moral considerations in political judgment is conditional on knowledge and exposure to political discourse. The factors that are expected to be related to references to moral foundations include \textit{political knowledge}, which was measured as the sum of correct answers to factual knowledge questions. The analyses also investigate the effect of \textit{political media exposure} and the frequency of \textit{political discussions} with friends and family members. Here, the analyses do not only examine the influence on individual foundations, but also consider whether these factors influence \textit{general} moral reasoning. This latter variable is measured as the sum of individual MFT scores across all dimensions (rescaled to unit variance after summation), which can be interpreted as a aggregate measure of how much respondents emphasize any moral consideration in their responses. Whenever appropriate, independent variables were rescaled to range from 0 to 1. Figure~\ref{fig:app_desc} in the appendix provides histograms of all independent variables included in different stages of the analyses.
%
%
%\subsection{MFT and Feeling Thermometers}
%
%\begin{figure}[ht]\centering
%\includegraphics{../calc/fig/ols_feel.pdf}
%\caption{Change in predicted feeling thermometer differential when MFT score is increased from its minimum (no overlap between dictionary and response) by one standard deviation, holding all other control variables constant at their respective means (along with 95\% confidence intervals). Positive values indicate that respondents who emphasized the respective foundation evaluated the Democratic candidate/party more favorably than the Republican candidate/party, and vice versa. Estimates are based on a single OLS model (using robust standard errors) including MFT scores for each foundation and gray triangles indicate estimates while additionally controlling for party identification. The dimension of purity/sanctity was omitted due to its low general prevalence in individual attitude expressions. Additional control variables include church attendance, education, age, sex, race, survey mode, response length, and the Wordsum vocabulary score. Full model results are displayed in the appendix, Table~\ref{tab:ols_feel}.
%}\label{fig:ols_feel}
%\end{figure}
%
%As a first step, we examine the relationship of moral reasoning and attitudes towards political parties and candidates. Figure~\ref{fig:ols_feel} presents OLS estimates where feeling thermometer differentials between the Republican and the Democratic party (left panel) and between both Presidential candidates (right panel) are regressed on MFT scores for all moral foundations (including the control variables discussed above). Positive values indicate more favorable evaluations for the Democratic candidate or party and negative values indicate more favorable evaluations of the Republican candidate or party. The patterns are largely consistent with the previous results on ideological differences. Individuals who emphasize considerations related to harm/care, and fairness/reciprocity evaluate the Democratic party/candidate on average about 3 points higher than the Republican party/candidate (on a 100 point scale). On the other hand, if individuals emphasized the ingroup/loyalty dimension, they reported stronger preferences for the Republican party/candidate. Most of these effects are robust after controlling for individual party identification. Thus, in both analyses in Figure~\ref{fig:ols_feel}, we observe sizable and significant effects for the influence of moral reasoning. Interestingly, mentioning terms that belong to the authority/respect dimension appears to increase favorability towards the democratic party and candidate, which contradicts MFT. However, the effect disappears once party identification is controlled for.
%
%
%\clearpage
%\section{Additional Descriptive Information}\label{app:oview}
%\renewcommand\thefigure{\thesection.\arabic{figure}}
%\renewcommand\thetable{\thesection.\arabic{table}}
%\setcounter{figure}{0}
%\setcounter{table}{0}
%
%% latex table generated in R 3.3.3 by xtable 1.8-2 package
% Fri Mar 24 00:46:38 2017
\begin{table}[ht]
\centering
\caption{Missing open-ended responses} 
\label{tab:app_mis}
\begin{tabular}{lcc}
  \hline
 & N & Percent \\ 
  \hline
Spanish Interview & 228 & 3.86 \\ 
  No/Short Responses & 655 & 11.08 \\ 
   \hline
\end{tabular}
\end{table}

%
%\begin{figure}[h]\centering
%\includegraphics[width=\textwidth]{../calc/fig/app_wc.pdf}
%\caption{Histograms displaying the distribution of individual response lengths in number of words for each respective item category. Dotted lines indicate the average response length.}\label{fig:appB2num}
%\end{figure}
%
%\begin{figure}[h]\centering
%\includegraphics[width=\textwidth]{../calc/fig/app_desc.pdf}
%\caption{Histograms for variables included in analyses.}\label{fig:app_desc}
%\end{figure}
%
%\subsection{Proportion of MFT responses}
%
%Figure~\ref{fig:prop_ideol} presents a first descriptive overview of moral reasoning in open-ended responses. The figure displays the proportion of respondents who mentioned words that were included in the five different moral foundations dictionaries as well as their 95\% confidence intervals.\footnote{Note that the proportions are based on the subset of the sample that provided a response to at least one of the open-ended items, and for which the interview was held in English.} Since responses for each individual represent their likes and dislikes across all eight open-ended items, each proportion indicates the percentage of individuals who mentioned a signal word belonging to the respective moral foundation in any of his or her open-ended responses evaluating the parties or candidates.
%
%\begin{figure}[ht]\centering
%\includegraphics{../calc/fig/prop_mft.pdf}
%\caption{Proportion of respondents mentioning each of the moral foundations in any of their open-ended responses, along with 95\% confidence intervals. The first two foundations are often labeled individualizing foundations, which have been shown to be more prevalent among liberals, while the remaining ones are described as binding foundations, which are more prevalent among conservatives.}\label{fig:prop_ideol}
%\end{figure}
%% ADD Note
%% add note that first two are individualizing foundations, and the latter are binding foundations
%
%The moral foundation most frequently mentioned is harm/care: About 42\% of the respondents mentioned at least one word included in respective dictionary. The second most frequently mentioned moral foundation is authority/respect with about 37\%. The proportion of respondents emphasizing ingroup/respect of fairness reciprocity is slightly lower with about 29\% and 23\%, respectively. Purity/sanctity, on the other hand, was almost never mentioned by any of the respondents. This finding is surprising, since other studies found the foundation to be an important predictor of divisive political attitudes \citep{koleva2012tracing}. This result suggests that the terms contained in the purity/sanctity dictionary might be too uncommon in the context of politics and therefore not relevant for attitude expression. Due to the very rare mentioning of the purity/sanctity dimension, the subsequent analyses will concentrate on the remaining four moral foundations.\footnote{Unfortunately, this issue cannot not be properly addressed by relying on weighting scheme employed here. The weights can correct for some distortions due to individual ubiquitous terms in the dictionaries, but it cannot compensate for the fact that the purity dictionary as a whole contains mostly words that are never mentioned by respondents.} Subsequent analyses focusing on the dimension of purity/sanctity in open-ended survey responses might necessitate a revision of the moral foundation dictionary.
%
%Overall, Figure~\ref{fig:prop_ideol} shows that a substantial proportion of individuals evokes moral considerations when describing their political attitudes even when they are not explicitly asked about morality. However, we are ultimately interested in ideological differences in the emphasis of moral foundations in open-ended responses. As such, we now turn to a more in-depth analysis of the MFT scores as measures of moral reasoning.
%
%\subsection{Media content analyses}
%
%\begin{figure}[ht]\centering
%\includegraphics[width=\textwidth]{../calc/fig/media_desc.pdf}
%\caption{MFT scores for media sources during 2012 U.S. Presidential campaign. Articles and scripts were selected if they mentioned either presidential candidate during the last month of the campaign (October). Contents were retrieved in full text from Lexis-Nexis (except for the Wall Street Journal, which only provided abstracts). Each media source was analyzed using the same procedure described for open-ended responses (weighted proportion of signal word occurrence for each foundation, c.f., equation~[\ref{eq:tfidf}]). Scores were median-centered and rescaled to unit variance. The figure also displays 95\% confidence intervals, which are based on parametric bootstraps of the document feature matrix of the entire corpus (500 iterations).}\label{fig:media_desc}
%\end{figure}
%
%\begin{figure}[h]\centering
%\includegraphics[width=.67\textwidth]{../calc/fig/app_lidesc.pdf}
%\caption{Histograms for variables included in replication survey.}\label{fig:app_lidesc}
%\end{figure}
%
%
%\clearpage
%\section{Additional Model Results and Robustness Checks}\label{app:robust}
%\renewcommand\thefigure{\thesection.\arabic{figure}}
%\renewcommand\thetable{\thesection.\arabic{table}}
%\setcounter{figure}{0}
%\setcounter{table}{0}
%
%
%\begin{figure}[h]\centering
%\includegraphics{../calc/fig/tobit_part.pdf}
%\caption{Change in predicted overall reliance on moral foundations depending on previous turnout and non-conventional forms of participation (protest, petitions, campaign buttons, letter to congressmen/senator). The plot shows differences in predicted probabilities of mentioning any moral foundation (left panel) as well as in the summed MFT scores given that any foundation was mentioned (right panel), if a respondent engaged in the respective form of participation (vs. not) holding all other variables constant at their respective means (along with 95\% confidence intervals). Positive values indicate higher probability of mentioning, or stronger emphasis on moral foundations. Estimates are based on Tobit models and gray triangles indicate estimates while additionally controlling for the remaining variables presented in the figure. Full model results are displayed in the appendix, Table~\ref{tab:tobit_part}.
%}\label{fig:tobit_part}
%\end{figure}
%
%
%\begin{figure}[ht]\centering
%\includegraphics[scale=.8]{../calc/fig/tobit_ideol_difdif.pdf}
%\caption{Change in effect of ideology on emphasis of each moral foundation moderated by political knowledge, media exposure, and frequency of political discussions (difference-in-difference). The plot shows how the difference between liberals and conservatives in predicted probabilities to mention each moral foundation, as well as the respective MFT scores, change if each of the independent variables is increased from its minimum to its maximum value holding control variables constant at their respective means (along with 95\% confidence intervals). Positive values indicate that liberals are more likely to mention a specific moral foundation if they score high on the moderating variable (knowledge, exposure, discussions, previous turnout, protest behavior), and vice versa. Estimates are based on individual Tobit models for each foundation and gray triangles indicate estimates while controlling for all remaining variables displayed in the figure. Full model results are displayed in Tables~\ref{tab:tobit_ideol_know}, \ref{tab:tobit_ideol_media}, \ref{tab:tobit_ideol_disc}, and \ref{tab:tobit_ideol_difdif}.
%}\label{fig:tobit_ideol_difdif}
%\end{figure}
%
%
%A related concern might be the question whether the content analysis of media sources using the dictionary is able to capture overall levels of moralization in news reporting. Luckily, a study reported in \citet{feinberg2013moral} included manual coding a selection of newspaper articles on environmental issues to capture whether they use rhetoric grounded in each of the moral domains. Their coding therefore focuses on the same foundations without utilizing the dictionary. I computed a general moralization variable by summing the scores used in \citet{feinberg2013moral} and compared them to the MFT scores based on the procedures outlined above.\footnote{I am indebted to the authors for providing the data.}
%
%\begin{figure}[ht]\centering
%\includegraphics{../calc/fig/feinberg_general.pdf}
%\caption{Validity check based on the data from \citet{feinberg2013moral}.}\label{fig:ols_feinberg}
%\end{figure}
%
%Figure~\ref{fig:ols_feinberg} presents the correlation of general moralization in each article based on the manual coding in \citet{feinberg2013moral} compared to the dictionary method used in the analyses presented here. While the correlation is far from being perfect, the weighted dictionary method clearly captures some of the same variance as manual assessments of the emphasis on moral foundations.
%

%\clearpage
%\section{Tables of Model Estimates}\label{app:tables}
%\renewcommand\thefigure{\thesection.\arabic{figure}}
%\renewcommand\thetable{\thesection.\arabic{table}}
%\setcounter{figure}{0}
%\setcounter{table}{0}
%
%
%\subsection*{Ideological Differences in Moral Reasoning}
%% latex table generated in R 3.3.2 by xtable 1.8-2 package
% Mon Feb 27 15:07:14 2017
\begin{table}[ht]
\centering
\caption{Tobit models predicting MFT score for each foundation based 
           on ideology. Positive coefficients indicate stronger emphasis on the respective 
           foundation. Standard errors in parentheses. Estimates are used for Figure 
           \ref{fig:tobit_ideol} in the main text.} 
\label{tab:tobit_ideol}
\begingroup\footnotesize
\begin{tabular}{lcccc}
  \hline
Variable & Harm & Fairness & Ingroup & Authority \\ 
  \hline
Ideology (Conservative) & -0.308 & -0.697 &  0.367 & -0.133 \\ 
   & (0.08) & (0.143) & (0.116) & (0.091) \\ 
  Ideology (Moderate) & -0.135 & -0.512 &  0.099 & -0.060 \\ 
   & (0.081) & (0.146) & (0.121) & (0.093) \\ 
  Church Attendance & -0.063 &  0.110 &  0.247 & -0.123 \\ 
   & (0.092) & (0.167) & (0.132) & (0.105) \\ 
  Education (College Degree) & -0.093 &  0.236 &  0.308 &  0.106 \\ 
   & (0.07) & (0.125) & (0.099) & (0.079) \\ 
  Age &  0.002 &  0.001 & -0.007 &  0.003 \\ 
   & (0.002) & (0.004) & (0.003) & (0.002) \\ 
  Sex (Female) &  0.134 &  0.078 & -0.244 & -0.094 \\ 
   & (0.063) & (0.114) & (0.091) & (0.072) \\ 
  Race (African American) &  0.045 & -0.125 & -0.216 &  0.339 \\ 
   & (0.091) & (0.166) & (0.135) & (0.101) \\ 
  Word Count (log) &  0.363 &  0.527 &  0.773 &  0.584 \\ 
   & (0.033) & (0.061) & (0.051) & (0.039) \\ 
  Wordsum Score &  0.561 &  0.660 &  0.603 &  0.325 \\ 
   & (0.166) & (0.302) & (0.242) & (0.188) \\ 
  Survey Mode (Online) & -0.039 &  0.205 &  0.149 &  0.301 \\ 
   & (0.076) & (0.138) & (0.11) & (0.087) \\ 
  Intercept & -2.503 & -4.742 & -4.860 & -3.619 \\ 
   & (0.193) & (0.363) & (0.297) & (0.227) \\ 
  log(Sigma) &  0.553 &  1.025 &  0.869 &  0.685 \\ 
   & (0.021) & (0.027) & (0.023) & (0.02) \\ 
   \hline
N & 4684 & 4684 & 4684 & 4684 \\ 
  Log-Likelihood & -4923 & -3961 & -4568 & -5045 \\ 
   \hline
\end{tabular}
\endgroup
\end{table}

%
%\clearpage
%\subsection*{The Political Relevance of Moral Reasoning}
%% latex table generated in R 3.3.0 by xtable 1.8-2 package
% Thu Nov 10 12:04:38 2016
\begin{table}[h]
\centering
\caption{OLS models predicting feeling thermometer differentials based on
           MFT score for each foundation. Positive coefficients indicate more favorable evaluation 
           of Democratic candidate/party than the Republican candidate/party, and vice versa. 
           Standard errors in parentheses. Estimates are used for Figure \ref{fig:ols_feel} 
           in the main text.} 
\label{tab:ols_feel}
\begingroup\footnotesize
\begin{tabular}{lcccc}
  \hline
Variable & Party (1) & Party (2) & Cand. (1) & Cand. (2) \\ 
  \hline
Harm &   2.454 &   0.874 &   2.632 &   0.955 \\ 
   & (0.716) & (0.5) & (0.857) & (0.643) \\ 
  Fairness &   1.831 &   0.691 &   3.091 &   1.790 \\ 
   & (0.623) & (0.435) & (0.748) & (0.56) \\ 
  Ingroup &  -2.911 &  -0.899 &  -4.006 &  -1.771 \\ 
   & (0.646) & (0.452) & (0.777) & (0.583) \\ 
  Authority &   2.249 &   0.524 &   2.251 &   0.366 \\ 
   & (0.669) & (0.467) & (0.795) & (0.596) \\ 
  PID (Democrat) &  &  44.597 &  &  47.207 \\ 
   &  & (1.073) &  & (1.375) \\ 
  PID (Republican) &  & -44.710 &  & -52.277 \\ 
   &  & (1.189) &  & (1.527) \\ 
  Church Attendance & -27.678 & -11.450 & -35.906 & -17.641 \\ 
   & (1.821) & (1.296) & (2.181) & (1.665) \\ 
  Education (College Degree) &   0.284 &   1.300 &   1.335 &   2.515 \\ 
   & (1.464) & (1.023) & (1.757) & (1.317) \\ 
  Age &  -0.107 &  -0.119 &  -0.306 &  -0.315 \\ 
   & (0.039) & (0.028) & (0.047) & (0.035) \\ 
  Sex (Female) &   7.461 &   2.926 &   9.270 &   4.373 \\ 
   & (1.28) & (0.897) & (1.532) & (1.152) \\ 
  Race (African American) &  52.954 &  20.981 &  63.129 &  28.209 \\ 
   & (1.739) & (1.294) & (2.08) & (1.659) \\ 
  Word Count (log) &   2.317 &   1.111 &   1.757 &   0.350 \\ 
   & (0.637) & (0.445) & (0.763) & (0.572) \\ 
  Wordsum Score &  -0.851 &   2.547 &   0.580 &   4.105 \\ 
   & (3.298) & (2.309) & (3.956) & (2.971) \\ 
  Survey Mode (Online) &  -5.828 &  -1.975 &  -8.689 &  -4.460 \\ 
   & (1.511) & (1.06) & (1.809) & (1.362) \\ 
  Intercept &   8.129 &   4.584 &  21.428 &  19.166 \\ 
   & (3.352) & (2.389) & (4.009) & (3.061) \\ 
   \hline
N & 5135 & 5123 & 5151 & 5140 \\ 
  R-squared (adj.) & 0.211 & 0.617 & 0.224 & 0.565 \\ 
   \hline
\end{tabular}
\endgroup
\end{table}

%% latex table generated in R 3.3.0 by xtable 1.8-2 package
% Thu Nov 10 12:04:38 2016
\begin{table}[ht]
\centering
\caption{Logit models predicting democratic vote choice based on
           MFT score for each foundation. Positive coefficients indicate higher likelihood
           to vote for the Democratic candidate than the Republican candidate. Standard errors 
           in parentheses. Estimates are used for Figure \ref{fig:logit_vote} in the main text.} 
\label{tab:logit_vote}
\begingroup\footnotesize
\begin{tabular}{lcc}
  \hline
Variable & (1) & (2) \\ 
  \hline
Harm &  0.263 &  0.242 \\ 
   & (0.064) & (0.091) \\ 
  Fairness &  0.198 &  0.170 \\ 
   & (0.055) & (0.07) \\ 
  Ingroup & -0.176 & -0.072 \\ 
   & (0.042) & (0.05) \\ 
  Authority &  0.071 &  0.013 \\ 
   & (0.041) & (0.056) \\ 
  PID (Democrat) &  &  2.570 \\ 
   &  & (0.133) \\ 
  PID (Republican) &  & -2.636 \\ 
   &  & (0.15) \\ 
  Church Attendance & -1.637 & -1.390 \\ 
   & (0.112) & (0.155) \\ 
  Education (College Degree) &  0.175 &  0.374 \\ 
   & (0.084) & (0.117) \\ 
  Age & -0.009 & -0.016 \\ 
   & (0.002) & (0.003) \\ 
  Sex (Female) &  0.259 &  0.131 \\ 
   & (0.077) & (0.105) \\ 
  Race (African American) &  4.239 &  3.261 \\ 
   & (0.262) & (0.286) \\ 
  Word Count (log) &  0.108 &  0.049 \\ 
   & (0.039) & (0.053) \\ 
  Wordsum Score & -0.038 &  0.070 \\ 
   & (0.206) & (0.282) \\ 
  Survey Mode (Online) & -0.361 & -0.382 \\ 
   & (0.094) & (0.128) \\ 
  Intercept &  0.537 &  0.855 \\ 
   & (0.217) & (0.294) \\ 
   \hline
N & 3827 & 3819 \\ 
  Log-Likelihood & -2023 & -1192 \\ 
   \hline
\end{tabular}
\endgroup
\end{table}

%
%\clearpage
%\subsection*{The Conditionality of Moral Reasoning}
%% latex table generated in R 3.3.0 by xtable 1.8-2 package
% Thu Nov  3 16:40:24 2016
\begin{table}[ht]
\centering
\caption{Tobit models predicting overall reliance on moral foundations
           (sum of MFT scores) based on political knowledge, media exposure, and frequency of 
           political discussions. Positive coefficients indicate stronger emphasis on any foundation.
           Standard errors in parentheses. Estimates are used for Figure \ref{fig:tobit_learn} in 
           the main text.} 
\label{tab:tobit_learn}
\begingroup\footnotesize
\begin{tabular}{lcccc}
  \hline
Variable & (1) & (2) & (3) & (4) \\ 
  \hline
Political Knowledge &  0.260 &  &  &  0.236 \\ 
   & (0.098) &  &  & (0.103) \\ 
  Political Media Exposure &  &  0.369 &  &  0.272 \\ 
   &  & (0.088) &  & (0.095) \\ 
  Political
Discussions &  &  &  0.263 &  0.202 \\ 
   &  &  & (0.068) & (0.07) \\ 
  Church Attendance & -0.020 & -0.022 & -0.006 & -0.010 \\ 
   & (0.052) & (0.053) & (0.055) & (0.055) \\ 
  Education (College Degree) &  0.079 &  0.079 &  0.097 &  0.070 \\ 
   & (0.043) & (0.042) & (0.044) & (0.044) \\ 
  Age & -0.001 & -0.002 &  0.000 & -0.002 \\ 
   & (0.001) & (0.001) & (0.001) & (0.001) \\ 
  Sex (Female) &  0.018 &  0.016 &  0.017 &  0.041 \\ 
   & (0.037) & (0.037) & (0.038) & (0.039) \\ 
  Race (African American) &  0.109 &  0.088 &  0.082 &  0.091 \\ 
   & (0.05) & (0.05) & (0.052) & (0.052) \\ 
  Word Count (log) &  0.099 &  0.097 &  0.090 &  0.080 \\ 
   & (0.019) & (0.018) & (0.019) & (0.02) \\ 
  Wordsum Score &  0.281 &  0.353 &  0.323 &  0.257 \\ 
   & (0.1) & (0.096) & (0.1) & (0.104) \\ 
  Survey Mode (Online) &  0.042 &  0.043 &  0.099 &  0.061 \\ 
   & (0.044) & (0.044) & (0.046) & (0.047) \\ 
  Intercept & -0.401 & -0.380 & -0.356 & -0.446 \\ 
   & (0.1) & (0.097) & (0.101) & (0.105) \\ 
  log(Sigma) &  0.209 &  0.209 &  0.213 &  0.212 \\ 
   & (0.014) & (0.014) & (0.014) & (0.014) \\ 
   \hline
N & 5173 & 5164 & 4834 & 4827 \\ 
  Log-Likelihood & -7132 & -7117 & -6687 & -6672 \\ 
   \hline
\end{tabular}
\endgroup
\end{table}

%% latex table generated in R 3.3.3 by xtable 1.8-2 package
% Sun Mar 19 13:46:11 2017
\begin{table}[ht]
\centering
\caption{Tobit models predicting MFT score for each foundation based 
           on political knowledge (mean-centered) and ideology. Positive coefficients indicate stronger 
           emphasis on the respective foundation. Standard errors in parentheses. Estimates are used 
           for Figure \ref{fig:tobit_ideol_know} in the main text.} 
\label{tab:tobit_ideol_know}
\begingroup\footnotesize
\begin{tabular}{lcccc}
  \hline
Variable & Harm & Fairness & Ingroup & Authority \\ 
  \hline
Political Knowledge &  0.795 & -0.236 & -0.037 &  0.882 \\ 
   & (0.299) & (0.472) & (0.406) & (0.312) \\ 
  Ideology (Conservative) & -0.325 & -0.866 &  0.340 & -0.092 \\ 
   & (0.092) & (0.152) & (0.123) & (0.096) \\ 
  Knowledge * Conservative & -0.984 &  0.780 &  0.551 & -0.596 \\ 
   & (0.388) & (0.632) & (0.511) & (0.4) \\ 
  Ideology (Moderate) & -0.190 & -0.730 &  0.044 & -0.003 \\ 
   & (0.091) & (0.15) & (0.125) & (0.095) \\ 
  Knowledge * Moderate & -0.503 &  0.653 &  0.136 & -1.077 \\ 
   & (0.405) & (0.667) & (0.552) & (0.417) \\ 
  Church Attendance &  0.014 &  0.066 &  0.255 & -0.105 \\ 
   & (0.103) & (0.169) & (0.134) & (0.105) \\ 
  Education (College Degree) & -0.129 &  0.280 &  0.328 &  0.078 \\ 
   & (0.079) & (0.128) & (0.102) & (0.08) \\ 
  Age &  0.000 &  0.001 & -0.008 &  0.002 \\ 
   & (0.002) & (0.004) & (0.003) & (0.002) \\ 
  Sex (Female) &  0.107 &  0.139 & -0.204 & -0.088 \\ 
   & (0.071) & (0.118) & (0.094) & (0.073) \\ 
  Race (African American) &  0.123 & -0.033 & -0.236 &  0.353 \\ 
   & (0.101) & (0.169) & (0.139) & (0.102) \\ 
  Word Count (log) &  0.410 &  0.601 &  0.751 &  0.492 \\ 
   & (0.041) & (0.068) & (0.055) & (0.042) \\ 
  Wordsum Score &  0.659 &  0.721 &  0.547 &  0.221 \\ 
   & (0.193) & (0.321) & (0.256) & (0.197) \\ 
  Survey Mode (Online) & -0.054 &  0.294 &  0.130 &  0.266 \\ 
   & (0.085) & (0.141) & (0.112) & (0.088) \\ 
  Intercept & -2.575 & -4.962 & -4.664 & -3.133 \\ 
   & (0.236) & (0.403) & (0.325) & (0.246) \\ 
  log(Sigma) &  0.669 &  1.045 &  0.884 &  0.683 \\ 
   & (0.02) & (0.026) & (0.022) & (0.02) \\ 
   \hline
N & 4489 & 4489 & 4489 & 4489 \\ 
  Log-Likelihood & -5144 & -3983 & -4579 & -5025 \\ 
   \hline
\end{tabular}
\endgroup
\end{table}

%% latex table generated in R 3.3.0 by xtable 1.8-2 package
% Thu Nov  3 16:40:24 2016
\begin{table}[ht]
\centering
\caption{Tobit models predicting MFT score for each foundation based 
           on political media exposure (mean-centered) and ideology. Positive coefficients indicate 
           stronger emphasis on the respective foundation. Standard errors in parentheses. Estimates 
           are used for Figure \ref{fig:tobit_ideol_media} in the main text.} 
\label{tab:tobit_ideol_media}
\begingroup\footnotesize
\begin{tabular}{lcccc}
  \hline
Variable & Harm & Fairness & Ingroup & Authority \\ 
  \hline
Political Media Exposure &  0.498 &  0.114 &  0.684 &  0.772 \\ 
   & (0.245) & (0.46) & (0.394) & (0.302) \\ 
  Ideology (Conservative) & -0.223 & -0.732 &  0.379 & -0.126 \\ 
   & (0.075) & (0.146) & (0.117) & (0.093) \\ 
  Media * Conservative & -0.514 &  0.810 & -0.388 & -0.166 \\ 
   & (0.317) & (0.606) & (0.491) & (0.387) \\ 
  Ideology (Moderate) & -0.125 & -0.507 &  0.123 & -0.034 \\ 
   & (0.076) & (0.146) & (0.122) & (0.093) \\ 
  Media * Moderate & -0.280 &  0.038 & -0.512 & -0.691 \\ 
   & (0.326) & (0.629) & (0.525) & (0.402) \\ 
  Church Attendance & -0.048 &  0.113 &  0.250 & -0.125 \\ 
   & (0.086) & (0.167) & (0.132) & (0.105) \\ 
  Education (College Degree) & -0.078 &  0.225 &  0.292 &  0.077 \\ 
   & (0.066) & (0.126) & (0.1) & (0.079) \\ 
  Age & -0.003 & -0.001 & -0.009 &  0.000 \\ 
   & (0.002) & (0.004) & (0.003) & (0.002) \\ 
  Sex (Female) &  0.175 &  0.101 & -0.233 & -0.070 \\ 
   & (0.059) & (0.115) & (0.092) & (0.072) \\ 
  Race (African American) &  0.014 & -0.120 & -0.219 &  0.322 \\ 
   & (0.085) & (0.167) & (0.135) & (0.102) \\ 
  Word Count (log) &  0.309 &  0.520 &  0.766 &  0.577 \\ 
   & (0.031) & (0.061) & (0.051) & (0.039) \\ 
  Wordsum Score &  0.553 &  0.651 &  0.620 &  0.320 \\ 
   & (0.154) & (0.302) & (0.241) & (0.188) \\ 
  Survey Mode (Online) & -0.112 &  0.190 &  0.133 &  0.284 \\ 
   & (0.071) & (0.139) & (0.11) & (0.087) \\ 
  Intercept & -1.937 & -4.621 & -4.769 & -3.485 \\ 
   & (0.182) & (0.372) & (0.305) & (0.232) \\ 
  log(Sigma) &  0.500 &  1.024 &  0.867 &  0.684 \\ 
   & (0.02) & (0.027) & (0.023) & (0.02) \\ 
   \hline
N & 4678 & 4678 & 4678 & 4678 \\ 
  Log-Likelihood & -5106 & -3955 & -4561 & -5033 \\ 
   \hline
\end{tabular}
\endgroup
\end{table}

%% latex table generated in R 3.3.3 by xtable 1.8-2 package
% Sun Mar 19 13:46:11 2017
\begin{table}[ht]
\centering
\caption{Tobit models predicting MFT score for each foundation based 
           on political discussion frequency (mean-centered) and ideology. Positive coefficients 
           indicate stronger emphasis on the respective foundation. Standard errors in parentheses. 
           Estimates are used for Figure \ref{fig:tobit_ideol_disc} in the main text.} 
\label{tab:tobit_ideol_disc}
\begingroup\footnotesize
\begin{tabular}{lcccc}
  \hline
Variable & Harm & Fairness & Ingroup & Authority \\ 
  \hline
Political Discussion &  0.012 & -0.021 &  0.167 &  0.208 \\ 
   & (0.214) & (0.337) & (0.286) & (0.218) \\ 
  Ideology (Conservative) & -0.448 & -0.839 &  0.289 & -0.120 \\ 
   & (0.093) & (0.151) & (0.122) & (0.094) \\ 
  Discussion * Conservative &  0.141 &  0.929 &  0.735 &  0.273 \\ 
   & (0.28) & (0.444) & (0.36) & (0.281) \\ 
  Ideology (Moderate) & -0.251 & -0.706 &  0.038 & -0.077 \\ 
   & (0.093) & (0.152) & (0.126) & (0.095) \\ 
  Discussion * Moderate & -0.180 &  1.119 &  0.634 & -0.527 \\ 
   & (0.318) & (0.504) & (0.416) & (0.324) \\ 
  Church Attendance &  0.072 &  0.088 &  0.274 & -0.133 \\ 
   & (0.106) & (0.173) & (0.137) & (0.107) \\ 
  Education (College Degree) & -0.142 &  0.269 &  0.344 &  0.124 \\ 
   & (0.08) & (0.128) & (0.102) & (0.08) \\ 
  Age &  0.000 &  0.001 & -0.007 &  0.003 \\ 
   & (0.002) & (0.004) & (0.003) & (0.002) \\ 
  Sex (Female) &  0.081 &  0.172 & -0.213 & -0.079 \\ 
   & (0.073) & (0.118) & (0.095) & (0.073) \\ 
  Race (African American) &  0.069 & -0.037 & -0.256 &  0.335 \\ 
   & (0.104) & (0.172) & (0.14) & (0.103) \\ 
  Word Count (log) &  0.414 &  0.547 &  0.709 &  0.475 \\ 
   & (0.042) & (0.07) & (0.057) & (0.043) \\ 
  Wordsum Score &  0.769 &  0.822 &  0.544 &  0.326 \\ 
   & (0.193) & (0.318) & (0.253) & (0.193) \\ 
  Survey Mode (Online) & -0.032 &  0.357 &  0.214 &  0.275 \\ 
   & (0.087) & (0.144) & (0.114) & (0.089) \\ 
  Intercept & -2.606 & -4.862 & -4.536 & -3.130 \\ 
   & (0.238) & (0.402) & (0.323) & (0.243) \\ 
  log(Sigma) &  0.667 &  1.031 &  0.868 &  0.665 \\ 
   & (0.02) & (0.027) & (0.023) & (0.021) \\ 
   \hline
N & 4196 & 4196 & 4196 & 4196 \\ 
  Log-Likelihood & -4837 & -3724 & -4275 & -4712 \\ 
   \hline
\end{tabular}
\endgroup
\end{table}

%% latex table generated in R 3.3.0 by xtable 1.8-2 package
% Thu Nov 10 12:04:39 2016
\begin{table}[ht]
\centering
\caption{Tobit models predicting MFT score for each foundation based 
           on moral content of individual media environments. Positive coefficients 
           indicate stronger emphasis on the respective foundation. Standard errors in parentheses. 
           Estimates are used for Figure \ref{fig:tobit_cont} in the main text.} 
\label{tab:tobit_cont}
\begingroup\footnotesize
\begin{tabular}{lcccc}
  \hline
Variable & Harm & Fairness & Ingroup & Authority \\ 
  \hline
Media MFT score (harm) &  0.030 &  &  &  \\ 
   & (0.016) &  &  &  \\ 
  Media MFT score (fairness) &  &  0.051 &  &  \\ 
   &  & (0.024) &  &  \\ 
  Media MFT score (ingroup) &  &  &  0.001 &  \\ 
   &  &  & (0.029) &  \\ 
  Media MFT score (authority) &  &  &  &  0.001 \\ 
   &  &  &  & (0.017) \\ 
  Church Attendance & -0.128 & -0.029 &  0.357 & -0.115 \\ 
   & (0.078) & (0.152) & (0.124) & (0.097) \\ 
  Education (College Degree) & -0.063 &  0.239 &  0.345 &  0.110 \\ 
   & (0.063) & (0.122) & (0.099) & (0.079) \\ 
  Age & -0.002 &  0.002 & -0.004 &  0.001 \\ 
   & (0.002) & (0.003) & (0.003) & (0.002) \\ 
  Sex (Female) &  0.196 &  0.153 & -0.308 & -0.092 \\ 
   & (0.055) & (0.109) & (0.088) & (0.069) \\ 
  Race (African American) &  0.161 & -0.020 & -0.272 &  0.383 \\ 
   & (0.073) & (0.15) & (0.123) & (0.092) \\ 
  Word Count (log) &  0.304 &  0.560 &  0.798 &  0.588 \\ 
   & (0.028) & (0.058) & (0.049) & (0.037) \\ 
  Wordsum Score &  0.519 &  0.679 &  0.452 &  0.255 \\ 
   & (0.143) & (0.288) & (0.233) & (0.181) \\ 
  Survey Mode (Online) & -0.117 &  0.322 &  0.248 &  0.268 \\ 
   & (0.064) & (0.128) & (0.104) & (0.081) \\ 
  Intercept & -2.025 & -5.414 & -4.945 & -3.535 \\ 
   & (0.152) & (0.331) & (0.269) & (0.201) \\ 
  log(Sigma) &  0.481 &  1.011 &  0.879 &  0.691 \\ 
   & (0.019) & (0.026) & (0.022) & (0.019) \\ 
   \hline
N & 5173 & 5173 & 5173 & 5173 \\ 
  Log-Likelihood & -5650 & -4222 & -4936 & -5581 \\ 
   \hline
\end{tabular}
\endgroup
\end{table}

%
%\clearpage
%\subsection*{Examining Alternative Explanations}
%% latex table generated in R 3.3.3 by xtable 1.8-2 package
% Tue Mar 14 23:30:32 2017
\begin{table}[ht]
\centering
\caption{Tobit models predicting MFT score for each foundation based 
           on ideology (telephone survey replication). Positive coefficients indicate stronger emphasis on the respective 
           foundation. Standard errors in parentheses. Estimates are used for Figure 
           \ref{fig:tobit_ideol_lisurvey} in the main text.} 
\label{tab:tobit_ideol_lisurvey}
\begingroup\footnotesize
\begin{tabular}{lcccc}
  \hline
Variable & Harm & Fairness & Ingroup & Authority \\ 
  \hline
Ideology (Conservative) &  -2.427 & -3.501 &   1.767 & -1.002 \\ 
   & (1.094) & (1.198) & (1.121) & (0.809) \\ 
  Ideology (Moderate) &  -1.361 & -1.917 &  -1.412 & -1.030 \\ 
   & (0.87) & (0.86) & (1.095) & (0.687) \\ 
  Church Attendance &  -0.893 &  1.360 &   1.015 &  1.462 \\ 
   & (1.272) & (1.252) & (1.374) & (0.977) \\ 
  Education (College Degree) &   0.805 &  0.481 &   1.210 &  1.097 \\ 
   & (0.782) & (0.763) & (0.884) & (0.606) \\ 
  Age &  -0.002 &  0.045 &  -0.031 & -0.057 \\ 
   & (0.026) & (0.025) & (0.029) & (0.022) \\ 
  Sex (Female) &  -0.768 & -0.090 &  -0.710 & -0.410 \\ 
   & (0.769) & (0.755) & (0.852) & (0.587) \\ 
  Race (African American) &   0.578 &  1.558 &  -0.023 &  0.387 \\ 
   & (1.719) & (1.602) & (2.102) & (1.305) \\ 
  Word Count (log) &   2.498 &  0.272 &   1.569 &  0.739 \\ 
   & (0.667) & (0.482) & (0.646) & (0.4) \\ 
  Intercept & -11.666 & -7.644 & -10.270 & -3.853 \\ 
   & (2.773) & (2.254) & (2.803) & (1.655) \\ 
  log(Sigma) &   1.500 &  1.435 &   1.594 &  1.325 \\ 
   & (0.121) & (0.139) & (0.127) & (0.108) \\ 
   \hline
N & 395 & 395 & 395 & 395 \\ 
  Log-Likelihood & -224 & -192 & -218 & -266 \\ 
   \hline
\end{tabular}
\endgroup
\end{table}

%
%\clearpage
%\subsection*{Additional Models and Robustness Checks in Appendix}
%% latex table generated in R 3.3.0 by xtable 1.8-2 package
% Thu Nov 10 12:04:39 2016
\begin{table}[ht]
\centering
\caption{Tobit models predicting overall reliance on moral foundations
           (sum of MFT scores) based on political participation. Positive coefficients indicate 
           stronger emphasis on any foundation. Standard errors in parentheses. Estimates are 
           used for Figure \ref{fig:tobit_part} in the appendix.} 
\label{tab:tobit_part}
\begingroup\footnotesize
\begin{tabular}{lcccccc}
  \hline
Variable & (1) & (2) & (3) & (4) & (5) & (6) \\ 
  \hline
Voted in 2008 &  0.096 &  &  &  &  &  0.091 \\ 
   & (0.052) &  &  &  &  & (0.054) \\ 
  Protest &  &  0.076 &  &  &  & -0.002 \\ 
   &  & (0.076) &  &  &  & (0.08) \\ 
  Petition &  &  &  0.122 &  &  &  0.091 \\ 
   &  &  & (0.041) &  &  & (0.044) \\ 
  Button &  &  &  &  0.135 &  &  0.106 \\ 
   &  &  &  & (0.051) &  & (0.052) \\ 
  Letter &  &  &  &  &  0.090 &  0.040 \\ 
   &  &  &  &  & (0.048) & (0.051) \\ 
  Church Attendance & -0.032 &  0.001 &  0.001 & -0.007 & -0.004 & -0.023 \\ 
   & (0.053) & (0.055) & (0.055) & (0.055) & (0.055) & (0.055) \\ 
  Education (College Degree) &  0.085 &  0.103 &  0.095 &  0.108 &  0.095 &  0.086 \\ 
   & (0.043) & (0.044) & (0.044) & (0.044) & (0.044) & (0.045) \\ 
  Age & -0.001 &  0.000 &  0.000 &  0.000 &  0.000 & -0.001 \\ 
   & (0.001) & (0.001) & (0.001) & (0.001) & (0.001) & (0.001) \\ 
  Sex (Female) &  0.002 &  0.011 &  0.010 &  0.010 &  0.015 &  0.014 \\ 
   & (0.037) & (0.038) & (0.038) & (0.038) & (0.038) & (0.039) \\ 
  Race (African American) &  0.089 &  0.088 &  0.090 &  0.069 &  0.092 &  0.071 \\ 
   & (0.05) & (0.052) & (0.052) & (0.052) & (0.052) & (0.053) \\ 
  Word Count (log) &  0.103 &  0.105 &  0.097 &  0.103 &  0.101 &  0.091 \\ 
   & (0.018) & (0.019) & (0.019) & (0.019) & (0.019) & (0.02) \\ 
  Wordsum Score &  0.339 &  0.349 &  0.312 &  0.350 &  0.338 &  0.299 \\ 
   & (0.096) & (0.1) & (0.101) & (0.1) & (0.1) & (0.101) \\ 
  Survey Mode (Online) &  0.053 &  0.080 &  0.070 &  0.073 &  0.069 &  0.056 \\ 
   & (0.044) & (0.045) & (0.046) & (0.045) & (0.046) & (0.046) \\ 
  Intercept & -0.353 & -0.385 & -0.362 & -0.376 & -0.358 & -0.364 \\ 
   & (0.097) & (0.101) & (0.102) & (0.101) & (0.102) & (0.103) \\ 
  log(Sigma) &  0.209 &  0.215 &  0.215 &  0.214 &  0.214 &  0.213 \\ 
   & (0.014) & (0.014) & (0.014) & (0.014) & (0.014) & (0.014) \\ 
   \hline
N & 5157 & 4846 & 4833 & 4849 & 4847 & 4816 \\ 
  Log-Likelihood & -7108 & -6709 & -6688 & -6709 & -6709 & -6657 \\ 
   \hline
\end{tabular}
\endgroup
\end{table}

%% latex table generated in R 3.3.0 by xtable 1.8-2 package
% Thu Nov 10 12:04:39 2016
\begin{table}[ht]
\centering
\caption{Tobit models predicting MFT score for each foundation based 
           on political knowledge, media exposure, and discussion frequency (all mean-centered)
           as well as ideology. Positive coefficients indicate stronger emphasis on the respective
           foundation. Standard errors in parentheses. Estimates are used for Figure
           \ref{fig:tobit_ideol_difdif} in the appendix.} 
\label{tab:tobit_ideol_difdif}
\begingroup\footnotesize
\begin{tabular}{lcccc}
  \hline
Variable & Harm & Fairness & Ingroup & Authority \\ 
  \hline
Political Knowledge &  0.668 & -0.191 & -0.096 &  0.789 \\ 
   & (0.262) & (0.492) & (0.407) & (0.315) \\ 
  Political Media Exposure &  0.412 & -0.032 &  0.604 &  0.586 \\ 
   & (0.266) & (0.502) & (0.417) & (0.32) \\ 
  Political Discussion &  0.034 &  0.200 &  0.059 &  0.078 \\ 
   & (0.19) & (0.356) & (0.295) & (0.226) \\ 
  Ideology (Conservative) & -0.241 & -0.774 &  0.260 & -0.069 \\ 
   & (0.082) & (0.16) & (0.125) & (0.098) \\ 
  Knowledge * Conservative & -0.468 &  0.526 &  0.514 & -0.701 \\ 
   & (0.348) & (0.671) & (0.524) & (0.413) \\ 
  Media * Conservative & -0.657 &  0.429 & -0.777 & -0.160 \\ 
   & (0.353) & (0.675) & (0.534) & (0.418) \\ 
  Discussion * Conservative &  0.307 &  0.537 &  0.853 &  0.303 \\ 
   & (0.25) & (0.474) & (0.375) & (0.294) \\ 
  Ideology (Moderate) & -0.133 & -0.495 &  0.077 & -0.038 \\ 
   & (0.08) & (0.154) & (0.126) & (0.096) \\ 
  Knowledge * Moderate & -0.406 & -0.365 &  0.782 & -0.948 \\ 
   & (0.355) & (0.685) & (0.561) & (0.425) \\ 
  Media * Moderate & -0.077 &  0.013 & -0.594 & -0.352 \\ 
   & (0.356) & (0.69) & (0.562) & (0.427) \\ 
  Discussion * Moderate & -0.304 &  0.737 &  0.659 & -0.417 \\ 
   & (0.28) & (0.526) & (0.426) & (0.334) \\ 
  Church Attendance &  0.000 &  0.124 &  0.256 & -0.146 \\ 
   & (0.09) & (0.174) & (0.135) & (0.106) \\ 
  Education (College Degree) & -0.112 &  0.227 &  0.287 &  0.074 \\ 
   & (0.069) & (0.132) & (0.103) & (0.081) \\ 
  Age & -0.003 &  0.000 & -0.008 &  0.001 \\ 
   & (0.002) & (0.004) & (0.003) & (0.002) \\ 
  Sex (Female) &  0.179 &  0.126 & -0.213 & -0.041 \\ 
   & (0.063) & (0.121) & (0.095) & (0.074) \\ 
  Race (African American) &  0.020 & -0.136 & -0.213 &  0.347 \\ 
   & (0.089) & (0.175) & (0.139) & (0.103) \\ 
  Word Count (log) &  0.298 &  0.478 &  0.720 &  0.547 \\ 
   & (0.033) & (0.065) & (0.053) & (0.04) \\ 
  Wordsum Score &  0.431 &  0.740 &  0.476 &  0.287 \\ 
   & (0.168) & (0.33) & (0.258) & (0.199) \\ 
  Survey Mode (Online) & -0.116 &  0.248 &  0.184 &  0.243 \\ 
   & (0.075) & (0.148) & (0.115) & (0.09) \\ 
  Intercept & -1.800 & -4.629 & -4.505 & -3.366 \\ 
   & (0.199) & (0.406) & (0.324) & (0.247) \\ 
  log(Sigma) &  0.507 &  1.033 &  0.855 &  0.664 \\ 
   & (0.021) & (0.028) & (0.023) & (0.021) \\ 
   \hline
N & 4372 & 4372 & 4372 & 4372 \\ 
  Log-Likelihood & -4801 & -3722 & -4267 & -4713 \\ 
   \hline
\end{tabular}
\endgroup
\end{table}




\end{document}